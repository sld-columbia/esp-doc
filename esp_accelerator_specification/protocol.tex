\chapter{Accelerator Signal Level Protocol}

\section{DMA Transactions}

%------------------------------------------
\begin{wraptable}{r}[0.25in]{0.4\textwidth}
\centering
\small
\caption{Encoding of DMA size}\label{tab:hsize}
\begin{tabular}{|c|c|c|}
\hline
  \textbf{Encoding} & \textbf{Name} & \textbf{Bitwidth} \\
\hline
  000  & BYTE & 8 \\
\hline
  001  & HWORD & 16 \\
\hline
  010  & WORD & 32 \\
\hline
  011  & DWORD & 64 \\
\hline
\end{tabular}
\end{wraptable}
%------------------------------------------
{
The master of a direct-memory access (DMA) transaction is always an
accelerator. The accelerator initiates a DMA read transfer through the
\texttt{dma\_read\_ctrl} channel and a DMA write transfer through the
\texttt{dma\_write\_ctrl} channel. Tables~\ref{tab:hsize} and~\ref{tab:dma_ctrl} describe
the fields and the encoding of the two control channels.

DMA control channels are LIC that follow a simple protocol: when both {\it
  valid} and {\it ready} control signals are set, the value of the data bus is
sampled by the slave. From the accelerator view point {\it valid} and {\it
  ready} are independent and there should be no combinational path between the
two.

An ESP accelerator does not issue requests using physical addresses. The field
{\it index} of the control channels indicates an offset with respect to a
virtual memory region reserved for the accelerator. The ESP device driver
allocates this region in pages and generates a corresponding page table. The ESP
accelerator socket handles address translation, therefore the accelerator can
operate as if the reserved area was contiguous.
}

\begin{table}[h!]
\centering
\small
\caption{Description of the DMA control channels {\it dma\_read\_ctrl} and {\it dma\_write\_ctrl}.}\label{tab:dma_ctrl}
\begin{tabular}{|p{2.25in} p{0.75in}| p{3.25in} |}
\hline
  \textbf{Signal}         & \textbf{Driver} & \textbf{Description} \\
\hline
  \verb dma_[read|write]_ctrl_data_index  & accelerator & Offset of a DMA read
                                                          or write transaction
                                                          expressed as number of
                                                          beats. This offset is
                                                          used to compute the
                                                          starting address of
                                                          the transaction. \\
\hline
  \verb dma_[read|write]_ctrl_data_length & accelerator & Length of a DMA read
                                                          or write transaction
                                                          expressed as number of
                                                          beats.\\
\hline
  \verb dma_[read|write]_ctrl_data_size  & accelerator & Bitwidth of the data
                                                         token for the DMA
                                                         transaction. This signal
                                                         is used to correct the
                                                         NoC flits when
                                                         the processor
                                                         architecture follows
                                                         the {\it big endian}
                                                         convention to store
                                                         data in memory. This
                                                         signal follows the
                                                         encoding in
                                                         Table~\ref{tab:hsize}.\\
\hline
  \verb dma_[read|write]_ctrl_valid  & accelerator & Flag indicating that a new
                                                     DMA transaction
                                                     request. When set, all data
                                                     fields must be valid. This
                                                     flag must not depend
                                                     combinationally from the
                                                     corresponding {\it ready}
                                                     signal.\\
\hline
  \verb dma_[read|write]_ctrl_ready  & socket & Flag indicating that the ESP
                                                socket is ready to accept a new
                                                DMA request. This
                                                flag must not depend
                                                combinationally from the
                                                corresponding {\it valid}
                                                signal.\\
\hline
\end{tabular}
\end{table}

\pagebreak

For an accelerator with N-bits DMA interface (e.g. 64 bits), the physical
address in Bytes of a DMA transaction is computed by the ESP socket as follows:

\begin{equation}
  addr = walk\_accelerator\_ptable(index * N / 8)
\end{equation}

The user-level driver is responsible to prepare data in memory using the same
offsets used for DMA transfers by the accelerator. Offset calculation can be
defined at design time by hard-coding the logic to compute offsets in the accelerator.
Alternatively, offsets can be computed in software and configured at run
time through user-defined CSRs.

\begin{figure}[h!]
\begin{tikztimingtable}[%
    timing/dslope=0.1,
    timing/.style={x=5ex,y=2ex},
    x=5ex,
    timing/rowdist=3ex,
    timing/name/.style={font=\sffamily\scriptsize}
]
 \busref{clk}                                 & 30{c}                                                                                  \\
 %% read ctrl
 \busref[31:0]{dma\_read\_ctrl\_data\_index}  & u D{0x0}  4U 2D{0x10} 7.5U                                                             \\
 \busref[31:0]{dma\_read\_ctrl\_data\_length} & u D{2}  4U 2D{4} 7.5U                                                                  \\
 \busref[2:0]{dma\_read\_ctrl\_data\_size}    & u D{010}  4U 2D{011} 7.5U                                                              \\
 \busref{dma\_read\_ctrl\_valid}              & l H  4L 2H 7.5L                                                                        \\
 \slv{dma\_read\_ctrl\_ready}                 & [blue] l  H 4L ;[dotted]L; H 7.5L                                                      \\
 %% read chnl
 \slv[63:0]{dma\_read\_chnl\_data}            & [blue] u 3U D{$d_0, d_1$} D{$d_2, d_3$} 3U D{$d_4$} 2D{$d_5$} U D{$d_6$} D{$d_7$} 0.5U \\
 \slv{dma\_read\_chnl\_valid}                 & [blue] l L ;[dotted]2L; 2H  2L ;[dotted]L; 3H ;[dotted]1L; 2H l                        \\
 \busref{dma\_read\_chnl\_ready}              & l L ;[dotted]L; 3H 2L 2H ;[dotted]1L; 4H l                                             \\
\extracode
\begin{pgfonlayer}{background}
\begin{scope}[semitransparent ,semithick]
\vertlines[darkgray,dotted]{0.5,1.5 ,...,15.0}
\end{scope}
\end{pgfonlayer}
\end{tikztimingtable}
\label{wave:dma_read}\caption{Example of a DMA read transaction. Signals driven
  by the ESP socket are marked in blue.}
\end{figure}

A DMA transaction is initiated with a single beat transfer on the DMA control
channels. Once this transfer completes on the read control channel, the
accelerator waits for the socket to fetch the requested data by setting {\it
ready} high on the DMA read channel. A beat is successfully transferred any time
{\it ready} and {\it valid} are set high during the same cycle. There is no
restriction on the throughput of the transfer: the accelerator can apply
backpressure by de-asserting the {\it ready} signal at any time. However, the
accelerator must eventually complete the transaction by receiving exactly the
number of beats requested. Early termination will cause a deadlock condition of
the socket. Depending on the length of the transfer, deadlock can propagate to
the NoC and even to a memory tile.

\begin{figure}[h!]
\begin{tikztimingtable}[%
    timing/dslope=0.1,
    timing/.style={x=5ex,y=2ex},
    x=5ex,
    timing/rowdist=3ex,
    timing/name/.style={font=\sffamily\scriptsize}
]
 \busref{clk}                                  & 30{c}                                                                            \\
 %% write ctrl
 \busref[31:0]{dma\_write\_ctrl\_data\_index}  & u D{0x0}  5U 2D{0x10} 6.5U                                                       \\
 \busref[31:0]{dma\_write\_ctrl\_data\_length} & u D{2}  5U 2D{4} 6.5U                                                            \\
 \busref[2:0]{dma\_write\_ctrl\_data\_size}    & u D{010}  5U 2D{011} 6.5U                                                        \\
 \busref{dma\_write\_ctrl\_valid}              & l H  5L 2H 6.5L                                                                  \\
 \slv{dma\_write\_ctrl\_ready}                 & [blue] l  H 5L ;[dotted]L; H 6.5L                                                \\
 %% write chnl
 \busref[63:0]{dma\_write\_chnl\_data}         & u 2U 2D{$d_0, d_1$} D{$d_2, d_3$} 3U D{$d_4$} D{$d_5$} 2D{$d_6$} U D{$d_7$} 0.5U \\
 \busref{dma\_write\_chnl\_valid}              & l 1L ;[dotted]L; 3H  3L 4H ;[dotted]L; H l                                       \\
 \slv{dma\_write\_chnl\_ready}                 & [blue] l L ;[dotted]2L; 2H 3L 2H ;[dotted]1L; 3H l                               \\
\extracode
\begin{pgfonlayer}{background}
\begin{scope}[semitransparent ,semithick]
\vertlines[darkgray,dotted]{0.5,1.5 ,...,15.0}
\end{scope}
\end{pgfonlayer}
\end{tikztimingtable}
\label{wave:dma_write}\caption{Example of a DMA write transaction. Signals driven
  by the ESP socket are marked in blue.}
\end{figure}


Symmetrically, when a DMA read transfer is configured, the accelerator must
transfer the exact number of beats set with the {\it length} field. Data beats
are transferred through the DMA write channel by setting the {\it valid} flag
high when the corresponding data signal is valid. A beat is transferred if both
{\it valid} and {\it ready} are set during the same cycle. No restriction is
imposed on the throughput of the transfer. The accelerator must hold valid data
on the DMA write channel when the socket is not ready to sample it. This
condition may occur in case of contention for NoC links, or external memory
channels.

Figure~\ref{wave:dma_read} and~\ref{wave:dma_write} show two examples of DMA
read and DMA write transactions. Signals in blue are driven by the socket, while
signals in black are driven by the accelerator. Dotted lines indicate
back-pressure, which can be applied by either the accelerator or the socket.
