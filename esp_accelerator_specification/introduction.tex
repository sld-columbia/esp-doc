\chapter{Introduction}

This document describes the signal-level protocol specification of an ESP
accelerator. The guide is intended for RTL designers who wish to implement a
native ESP accelerator using a hardware-description language, such as
SystemVerilog, or VHDL. Any accelerator that complies with the protocol
specification described in this guide can be integrated in ESP and leverage all
platform services through the {\em ESP accelerator socket}.

This document does not describe the {\em ESP third-party accelerator flow}. The
latter enables the seamless integration of an existing accelerator IP leveraging
an ARM AMBA open standard interface.

\section{Conventions}

\begin{itemize}
  \item {\bf bitwidth:} number of bits. This is typically associated to a
    signal, or to a unit of data.
  \item {\bf token:} the unit of input or output data transferred between the
    accelerator and the ESP socket. The bitwidth of a token depends on the
    particular accelerator and may vary across different transactions over a bus
    or data channel.
  \item {\bf beat:} the unit of data transferred on a bus, or a  data
    channel. The bitwidth of one beat depends on the particular implementation
    of the accelerator (e.g. {\it dma32} or {\it dma64}) and not on the data
    type of the input or output token in a transaction. Therefore, for any given
    implementation of an ESP accelerator, the bitwidth of a beat is constant.
  \item {\bf flit:} the unit of data transferred over a network-on-chip
    (NoC). For ESP accelerators, the bitwidth of a flit is equal to the bitwidth
    of a beat plus two bits. These additional bits indicate if the flit is the
    head, part of the body, or the tail of a packet.
  \item {\bf packet:} a set of flits transferred in an ordered sequence across
    the NoC. Packets must have one header flit, one tail flit and as many body
    flits as necessary. Single-flit packets have just one flit with both {\it
      head} and {\it tail} bits set. A packet that is granted a link of the NoC
    will traverse such link from head to tail not interleaved with another
    packet.
  \item {\bf master:} a component that can initiate a transaction over a bus, or
    a NoC.
  \item {bf slave:} a component that servers a transaction initiated by a master.
  \item {\bf latency-insensitive channel (LIC):} a bundle of data wires and two
    control wires named {\it ready} and {\it valid}. During read transactions, the
    master drives the {\it ready} control signal, while the slave drives the
    data and the paired {\it valid} control signal. Roles are inverted for write
    transactions. A beat is transferred over a LIC when both {\it ready} and
    {\it valid} are set. Both master and slave have the ability to delay the
    transfer of a beat for as many cycles as necessary.
  \item {\bf CSR:} configuration and/or status register.
\end{itemize}

