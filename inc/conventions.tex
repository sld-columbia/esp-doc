\section{Conventions}

\begin{itemize}
  \item {\bf bitwidth:} number of bits. This is typically associated to a
    signal, or to a unit of data.
  \item {\bf token:} the unit of input or output data transferred between the
    accelerator and the ESP socket. The bitwidth of a token depends on the
    particular accelerator and may vary across different transactions over a bus
    or data channel.
  \item {\bf beat:} the unit of data transferred on a bus, or a  data
    channel. The bitwidth of one beat depends on the particular implementation
    of the accelerator (e.g. {\it dma32} or {\it dma64}) and not on the data
    type of the input or output token in a transaction. Therefore, for any given
    implementation of an ESP accelerator, the bitwidth of a beat is constant.
  \item {\bf flit:} the unit of data transferred over a network-on-chip
    (NoC). For ESP accelerators, the bitwidth of a flit is equal to the bitwidth
    of a beat plus two bits. These additional bits indicate if the flit is the
    head, part of the body, or the tail of a packet.
  \item {\bf packet:} a set of flits transferred in an ordered sequence across
    the NoC. Packets must have one header flit, one tail flit and as many body
    flits as necessary. Single-flit packets have just one flit with both {\it
      head} and {\it tail} bits set. A packet that is granted a link of the NoC
    will traverse such link from head to tail not interleaved with another
    packet.
  \item {\bf initiator} or {\bf master:} a component that can initiate a
    transaction over a bus, or a NoC.
  \item {\bf target} or {\bf slave:} a component that servers a transaction
    initiated by a master.
  \item {\bf latency-insensitive channel (LIC):} a bundle of data wires and two
    control wires named {\it ready} and {\it valid}. During read transactions,
    the master drives the {\it ready} control signal, while the slave drives the
    data and the paired {\it valid} control signal. Roles are inverted for write
    transactions. A beat is transferred over a LIC when both {\it ready} and
    {\it valid} are set. Both master and slave have the ability to delay the
    transfer of a beat for as many cycles as necessary.
  \item {\bf CSR:} configuration and/or status register.
  \item {\bf DMA:} the acronym for {\it direct-memory access}. When referring to
    an ESP accelerator, the term DMA refers to the mechanism used by the
    accelerator to access data in the system memory hierarchy. A DMA transaction
    initiated by an accelerator in ESP may be accessing external memory
    {\em directly} or by mediation of the ESP cache hierarchy. The selection is
    managed by software at run time and is transparent to the accelerator.
  \item {\bf PLM:} the accelerator's private local memory, composed of a set of
    SRAM bank groups customized for the accelerator's datapath.
\end{itemize}
